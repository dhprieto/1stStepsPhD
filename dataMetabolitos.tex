% Options for packages loaded elsewhere
\PassOptionsToPackage{unicode}{hyperref}
\PassOptionsToPackage{hyphens}{url}
%
\documentclass[
]{article}
\usepackage{amsmath,amssymb}
\usepackage{lmodern}
\usepackage{ifxetex,ifluatex}
\ifnum 0\ifxetex 1\fi\ifluatex 1\fi=0 % if pdftex
  \usepackage[T1]{fontenc}
  \usepackage[utf8]{inputenc}
  \usepackage{textcomp} % provide euro and other symbols
\else % if luatex or xetex
  \usepackage{unicode-math}
  \defaultfontfeatures{Scale=MatchLowercase}
  \defaultfontfeatures[\rmfamily]{Ligatures=TeX,Scale=1}
\fi
% Use upquote if available, for straight quotes in verbatim environments
\IfFileExists{upquote.sty}{\usepackage{upquote}}{}
\IfFileExists{microtype.sty}{% use microtype if available
  \usepackage[]{microtype}
  \UseMicrotypeSet[protrusion]{basicmath} % disable protrusion for tt fonts
}{}
\makeatletter
\@ifundefined{KOMAClassName}{% if non-KOMA class
  \IfFileExists{parskip.sty}{%
    \usepackage{parskip}
  }{% else
    \setlength{\parindent}{0pt}
    \setlength{\parskip}{6pt plus 2pt minus 1pt}}
}{% if KOMA class
  \KOMAoptions{parskip=half}}
\makeatother
\usepackage{xcolor}
\IfFileExists{xurl.sty}{\usepackage{xurl}}{} % add URL line breaks if available
\IfFileExists{bookmark.sty}{\usepackage{bookmark}}{\usepackage{hyperref}}
\hypersetup{
  pdftitle={descripionTablas.Rmd},
  hidelinks,
  pdfcreator={LaTeX via pandoc}}
\urlstyle{same} % disable monospaced font for URLs
\usepackage[margin=1in]{geometry}
\usepackage{color}
\usepackage{fancyvrb}
\newcommand{\VerbBar}{|}
\newcommand{\VERB}{\Verb[commandchars=\\\{\}]}
\DefineVerbatimEnvironment{Highlighting}{Verbatim}{commandchars=\\\{\}}
% Add ',fontsize=\small' for more characters per line
\usepackage{framed}
\definecolor{shadecolor}{RGB}{248,248,248}
\newenvironment{Shaded}{\begin{snugshade}}{\end{snugshade}}
\newcommand{\AlertTok}[1]{\textcolor[rgb]{0.94,0.16,0.16}{#1}}
\newcommand{\AnnotationTok}[1]{\textcolor[rgb]{0.56,0.35,0.01}{\textbf{\textit{#1}}}}
\newcommand{\AttributeTok}[1]{\textcolor[rgb]{0.77,0.63,0.00}{#1}}
\newcommand{\BaseNTok}[1]{\textcolor[rgb]{0.00,0.00,0.81}{#1}}
\newcommand{\BuiltInTok}[1]{#1}
\newcommand{\CharTok}[1]{\textcolor[rgb]{0.31,0.60,0.02}{#1}}
\newcommand{\CommentTok}[1]{\textcolor[rgb]{0.56,0.35,0.01}{\textit{#1}}}
\newcommand{\CommentVarTok}[1]{\textcolor[rgb]{0.56,0.35,0.01}{\textbf{\textit{#1}}}}
\newcommand{\ConstantTok}[1]{\textcolor[rgb]{0.00,0.00,0.00}{#1}}
\newcommand{\ControlFlowTok}[1]{\textcolor[rgb]{0.13,0.29,0.53}{\textbf{#1}}}
\newcommand{\DataTypeTok}[1]{\textcolor[rgb]{0.13,0.29,0.53}{#1}}
\newcommand{\DecValTok}[1]{\textcolor[rgb]{0.00,0.00,0.81}{#1}}
\newcommand{\DocumentationTok}[1]{\textcolor[rgb]{0.56,0.35,0.01}{\textbf{\textit{#1}}}}
\newcommand{\ErrorTok}[1]{\textcolor[rgb]{0.64,0.00,0.00}{\textbf{#1}}}
\newcommand{\ExtensionTok}[1]{#1}
\newcommand{\FloatTok}[1]{\textcolor[rgb]{0.00,0.00,0.81}{#1}}
\newcommand{\FunctionTok}[1]{\textcolor[rgb]{0.00,0.00,0.00}{#1}}
\newcommand{\ImportTok}[1]{#1}
\newcommand{\InformationTok}[1]{\textcolor[rgb]{0.56,0.35,0.01}{\textbf{\textit{#1}}}}
\newcommand{\KeywordTok}[1]{\textcolor[rgb]{0.13,0.29,0.53}{\textbf{#1}}}
\newcommand{\NormalTok}[1]{#1}
\newcommand{\OperatorTok}[1]{\textcolor[rgb]{0.81,0.36,0.00}{\textbf{#1}}}
\newcommand{\OtherTok}[1]{\textcolor[rgb]{0.56,0.35,0.01}{#1}}
\newcommand{\PreprocessorTok}[1]{\textcolor[rgb]{0.56,0.35,0.01}{\textit{#1}}}
\newcommand{\RegionMarkerTok}[1]{#1}
\newcommand{\SpecialCharTok}[1]{\textcolor[rgb]{0.00,0.00,0.00}{#1}}
\newcommand{\SpecialStringTok}[1]{\textcolor[rgb]{0.31,0.60,0.02}{#1}}
\newcommand{\StringTok}[1]{\textcolor[rgb]{0.31,0.60,0.02}{#1}}
\newcommand{\VariableTok}[1]{\textcolor[rgb]{0.00,0.00,0.00}{#1}}
\newcommand{\VerbatimStringTok}[1]{\textcolor[rgb]{0.31,0.60,0.02}{#1}}
\newcommand{\WarningTok}[1]{\textcolor[rgb]{0.56,0.35,0.01}{\textbf{\textit{#1}}}}
\usepackage{longtable,booktabs,array}
\usepackage{calc} % for calculating minipage widths
% Correct order of tables after \paragraph or \subparagraph
\usepackage{etoolbox}
\makeatletter
\patchcmd\longtable{\par}{\if@noskipsec\mbox{}\fi\par}{}{}
\makeatother
% Allow footnotes in longtable head/foot
\IfFileExists{footnotehyper.sty}{\usepackage{footnotehyper}}{\usepackage{footnote}}
\makesavenoteenv{longtable}
\usepackage{graphicx}
\makeatletter
\def\maxwidth{\ifdim\Gin@nat@width>\linewidth\linewidth\else\Gin@nat@width\fi}
\def\maxheight{\ifdim\Gin@nat@height>\textheight\textheight\else\Gin@nat@height\fi}
\makeatother
% Scale images if necessary, so that they will not overflow the page
% margins by default, and it is still possible to overwrite the defaults
% using explicit options in \includegraphics[width, height, ...]{}
\setkeys{Gin}{width=\maxwidth,height=\maxheight,keepaspectratio}
% Set default figure placement to htbp
\makeatletter
\def\fps@figure{htbp}
\makeatother
\setlength{\emergencystretch}{3em} % prevent overfull lines
\providecommand{\tightlist}{%
  \setlength{\itemsep}{0pt}\setlength{\parskip}{0pt}}
\setcounter{secnumdepth}{-\maxdimen} % remove section numbering
\ifluatex
  \usepackage{selnolig}  % disable illegal ligatures
\fi

\title{descripionTablas.Rmd}
\author{}
\date{\vspace{-2.5em}}

\begin{document}
\maketitle

\hypertarget{descripciuxf3n-datos}{%
\section{Descripción datos}\label{descripciuxf3n-datos}}

\hypertarget{compuestos-fenuxf3licos-procesado}{%
\subsection{Compuestos fenólicos
(Procesado?)}\label{compuestos-fenuxf3licos-procesado}}

Primero, se mide el efecto de agregar un endulzante determinado al zumo
y se van haciendo medidas de metabolitos en diferentes tiempos y
condiciones.Tenemos entonces dos factores que determinan el valor de los
mg/100 ml de zumo, que serían las condiciones de toma de la muestra y el
tipo de compuesto que se está midiendo.

Medimos cuatro compuestos:

\begin{itemize}
\tightlist
\item
  Derivados del Naringenin hexosido
\item
  Eriodictiol hexosa
\item
  Naringenin rutinosido
\item
  Hespertin rutinosido.
\end{itemize}

En primer lugar se extrae el área del cromatógrafo y se transforma en
mg/100 ml de zumo, con la fórmula siguiente:

\[((AREA-221,08)/18129)*610,19*(0,1)\]

Se repite la medición cada 15 días, en condiciones de luz y oscuridad, a
5º y a 25º celsius de temperatura.

El procesado estadístico realizado en el excel adjuntado consiste en
hacer la media y la desviación estándar de cada repetición hecha en las
mismas condiciones. Después se calculan los mg totales de compuestos
detectados por cada repetición, se hace la media y la SD, además de un
porcentaje de pérdida respecto a la primera medida. Se replica en R
(faltan los totales y la pérdida)

\hypertarget{procesado-y-ordenamiento}{%
\subsection{Procesado y ordenamiento:}\label{procesado-y-ordenamiento}}

\begin{Shaded}
\begin{Highlighting}[]
\CommentTok{\# Funcion de procesado de tablas}

\NormalTok{procesado }\OtherTok{\textless{}{-}} \ControlFlowTok{function}\NormalTok{(tabla, endulzante)\{}
  
\NormalTok{  tabla\_num }\OtherTok{\textless{}{-}}\NormalTok{ tabla[}\SpecialCharTok{{-}}\DecValTok{1}\NormalTok{]}
  
  \CommentTok{\# Se obtienen las medias de cada medida}
  
\NormalTok{  tabla\_mean }\OtherTok{\textless{}{-}} \FunctionTok{do.call}\NormalTok{(rbind, }
                        \FunctionTok{lapply}\NormalTok{(}\FunctionTok{seq}\NormalTok{(}\DecValTok{1}\NormalTok{, }\FunctionTok{nrow}\NormalTok{(tabla\_num), }\DecValTok{2}\NormalTok{), }\ControlFlowTok{function}\NormalTok{(i)\{}
\NormalTok{                          x }\OtherTok{\textless{}{-}}\NormalTok{ tabla\_num[ i}\SpecialCharTok{:}\NormalTok{(i }\SpecialCharTok{+} \DecValTok{1}\NormalTok{), , drop }\OtherTok{=} \ConstantTok{FALSE}\NormalTok{]}
\NormalTok{                          res }\OtherTok{\textless{}{-}} \FunctionTok{rbind}\NormalTok{(}\FunctionTok{colSums}\NormalTok{(x)}\SpecialCharTok{/}\DecValTok{2}\NormalTok{)}
\NormalTok{                          res}
\NormalTok{                        \}))}
  
  \FunctionTok{rownames}\NormalTok{(tabla\_mean) }\OtherTok{\textless{}{-}} \FunctionTok{unique}\NormalTok{(tabla}\SpecialCharTok{$}\NormalTok{Condiciones)}
  
  \CommentTok{\# La desviación estándar}
  
\NormalTok{  tabla\_sd }\OtherTok{\textless{}{-}} \FunctionTok{do.call}\NormalTok{(rbind, }
                      \FunctionTok{lapply}\NormalTok{(}\FunctionTok{seq}\NormalTok{(}\DecValTok{1}\NormalTok{, }\FunctionTok{nrow}\NormalTok{(tabla\_num), }\DecValTok{2}\NormalTok{), }\ControlFlowTok{function}\NormalTok{(i)\{}
\NormalTok{                        x }\OtherTok{\textless{}{-}}\NormalTok{ tabla\_num[ i}\SpecialCharTok{:}\NormalTok{(i }\SpecialCharTok{+} \DecValTok{1}\NormalTok{), , drop }\OtherTok{=} \ConstantTok{FALSE}\NormalTok{]}
\NormalTok{                        res }\OtherTok{\textless{}{-}} \FunctionTok{rbind}\NormalTok{(}\FunctionTok{apply}\NormalTok{(x, }\DecValTok{2}\NormalTok{, sd))}
\NormalTok{                        res}
\NormalTok{                      \}))}
  
  \CommentTok{\# Apilamos datos}
  
\NormalTok{  tabla\_1 }\OtherTok{\textless{}{-}} \FunctionTok{stack}\NormalTok{(}\FunctionTok{as.data.frame}\NormalTok{(tabla\_mean))}
\NormalTok{  tabla\_1}\SpecialCharTok{$}\NormalTok{Condiciones }\OtherTok{\textless{}{-}} \FunctionTok{rep}\NormalTok{(}\FunctionTok{unique}\NormalTok{(tabla}\SpecialCharTok{$}\NormalTok{Condiciones),}\DecValTok{4}\NormalTok{)}
  
\NormalTok{  tabla\_2 }\OtherTok{\textless{}{-}} \FunctionTok{stack}\NormalTok{(}\FunctionTok{as.data.frame}\NormalTok{(tabla\_sd))}
\NormalTok{  tabla\_2}\SpecialCharTok{$}\NormalTok{Condiciones }\OtherTok{\textless{}{-}} \FunctionTok{rep}\NormalTok{(}\FunctionTok{unique}\NormalTok{(tabla}\SpecialCharTok{$}\NormalTok{Condiciones),}\DecValTok{4}\NormalTok{)}
  
\NormalTok{  tabla\_total }\OtherTok{\textless{}{-}} \FunctionTok{merge}\NormalTok{(tabla\_1, tabla\_2, }\AttributeTok{by =} \FunctionTok{c}\NormalTok{(}\StringTok{"Condiciones"}\NormalTok{, }\StringTok{"ind"}\NormalTok{))}
  
\NormalTok{  tabla\_total}\SpecialCharTok{$}\NormalTok{Endulzante }\OtherTok{\textless{}{-}} \FunctionTok{rep}\NormalTok{(endulzante, }\FunctionTok{nrow}\NormalTok{(tabla\_total))}
  
\NormalTok{  dplyr}\SpecialCharTok{::}\FunctionTok{rename}\NormalTok{(tabla\_total, }\AttributeTok{mean =}\NormalTok{ values.x , }\AttributeTok{SD =}\NormalTok{ values.y, }\AttributeTok{Compuesto =}\NormalTok{ ind)}
  
\NormalTok{\}}

\CommentTok{\# {-}{-}{-}{-} Lectura y procesado de las tablas}

\NormalTok{fenSU }\OtherTok{\textless{}{-}} \FunctionTok{read.csv}\NormalTok{(}\StringTok{"data/Compuestos fenolicos SU VidaUtil Cronico\_1.csv"}\NormalTok{, }\AttributeTok{sep =} \StringTok{";"}\NormalTok{, }\AttributeTok{dec =} \StringTok{","}\NormalTok{)}
\NormalTok{fenSA }\OtherTok{\textless{}{-}} \FunctionTok{read.csv}\NormalTok{(}\StringTok{"data/Compuestos fenolicos SA VidaUtil Cronico\_1.csv"}\NormalTok{, }\AttributeTok{sep =} \StringTok{";"}\NormalTok{, }\AttributeTok{dec =} \StringTok{","}\NormalTok{)}
\NormalTok{fenST }\OtherTok{\textless{}{-}} \FunctionTok{read.csv}\NormalTok{(}\StringTok{"data/Compuestos fenolicos ST VidaUtil Cronico\_1.csv"}\NormalTok{, }\AttributeTok{sep =} \StringTok{";"}\NormalTok{, }\AttributeTok{dec =} \StringTok{","}\NormalTok{)}


\NormalTok{fenSU\_total }\OtherTok{\textless{}{-}} \FunctionTok{procesado}\NormalTok{(fenSU, }\StringTok{"SU"}\NormalTok{)}
\NormalTok{fenSA\_total }\OtherTok{\textless{}{-}} \FunctionTok{procesado}\NormalTok{(fenSA, }\StringTok{"SA"}\NormalTok{)}
\NormalTok{fenST\_total }\OtherTok{\textless{}{-}} \FunctionTok{procesado}\NormalTok{(fenST, }\StringTok{"ST"}\NormalTok{)}

\NormalTok{fenFlavTotal }\OtherTok{\textless{}{-}} \FunctionTok{rbind}\NormalTok{(fenSU\_total, fenSA\_total, fenST\_total)}

\NormalTok{skimr}\SpecialCharTok{::}\FunctionTok{skim}\NormalTok{(fenFlavTotal)}
\end{Highlighting}
\end{Shaded}

\begin{longtable}[]{@{}ll@{}}
\caption{Data summary}\tabularnewline
\toprule
& \\
\midrule
\endfirsthead
\toprule
& \\
\midrule
\endhead
Name & fenFlavTotal \\
Number of rows & 192 \\
Number of columns & 5 \\
\_\_\_\_\_\_\_\_\_\_\_\_\_\_\_\_\_\_\_\_\_\_\_ & \\
Column type frequency: & \\
character & 2 \\
factor & 1 \\
numeric & 2 \\
\_\_\_\_\_\_\_\_\_\_\_\_\_\_\_\_\_\_\_\_\_\_\_\_ & \\
Group variables & None \\
\bottomrule
\end{longtable}

\textbf{Variable type: character}

\begin{longtable}[]{@{}lrrrrrrr@{}}
\toprule
skim\_variable & n\_missing & complete\_rate & min & max & empty &
n\_unique & whitespace \\
\midrule
\endhead
Condiciones & 0 & 1 & 13 & 24 & 0 & 48 & 0 \\
Endulzante & 0 & 1 & 2 & 2 & 0 & 3 & 0 \\
\bottomrule
\end{longtable}

\textbf{Variable type: factor}

\begin{longtable}[]{@{}lrrlrl@{}}
\toprule
skim\_variable & n\_missing & complete\_rate & ordered & n\_unique &
top\_counts \\
\midrule
\endhead
Compuesto & 0 & 1 & FALSE & 4 & Der: 48, Eri: 48, Nar: 48, Hes: 48 \\
\bottomrule
\end{longtable}

\textbf{Variable type: numeric}

\begin{longtable}[]{@{}lrrrrrrrrrl@{}}
\toprule
skim\_variable & n\_missing & complete\_rate & mean & sd & p0 & p25 &
p50 & p75 & p100 & hist \\
\midrule
\endhead
mean & 0 & 1 & 1.64 & 1.47 & -0.04 & 0.56 & 1.35 & 1.79 & 5.98 &
▆▇▁▂▁ \\
SD & 0 & 1 & 0.03 & 0.07 & 0.00 & 0.00 & 0.00 & 0.01 & 0.57 & ▇▁▁▁▁ \\
\bottomrule
\end{longtable}

\hypertarget{pruebas-adicionales}{%
\subsection{Pruebas adicionales}\label{pruebas-adicionales}}

\hypertarget{normalidad}{%
\paragraph{Normalidad}\label{normalidad}}

\begin{Shaded}
\begin{Highlighting}[]
\FunctionTok{shapiro.test}\NormalTok{(fenFlavTotal}\SpecialCharTok{$}\NormalTok{mean)}
\end{Highlighting}
\end{Shaded}

\begin{verbatim}
## 
##  Shapiro-Wilk normality test
## 
## data:  fenFlavTotal$mean
## W = 0.85455, p-value = 1.432e-12
\end{verbatim}

\hypertarget{homocedasticidad}{%
\subsubsection{Homocedasticidad}\label{homocedasticidad}}

\begin{Shaded}
\begin{Highlighting}[]
\FunctionTok{var.test}\NormalTok{(}\AttributeTok{x =}\NormalTok{ fenFlavTotal[fenFlavTotal}\SpecialCharTok{$}\NormalTok{Endulzante }\SpecialCharTok{==} \StringTok{"SA"}\NormalTok{, }\StringTok{"mean"}\NormalTok{],}
         \AttributeTok{y =}\NormalTok{ fenFlavTotal[fenFlavTotal}\SpecialCharTok{$}\NormalTok{Endulzante }\SpecialCharTok{==} \StringTok{"SU"}\NormalTok{, }\StringTok{"mean"}\NormalTok{] )}
\end{Highlighting}
\end{Shaded}

\begin{verbatim}
## 
##  F test to compare two variances
## 
## data:  fenFlavTotal[fenFlavTotal$Endulzante == "SA", "mean"] and fenFlavTotal[fenFlavTotal$Endulzante == "SU", "mean"]
## F = 1.4891, num df = 63, denom df = 63, p-value = 0.1167
## alternative hypothesis: true ratio of variances is not equal to 1
## 95 percent confidence interval:
##  0.9046806 2.4511295
## sample estimates:
## ratio of variances 
##           1.489124
\end{verbatim}

\begin{Shaded}
\begin{Highlighting}[]
\FunctionTok{var.test}\NormalTok{(}\AttributeTok{x =}\NormalTok{ fenFlavTotal[fenFlavTotal}\SpecialCharTok{$}\NormalTok{Endulzante }\SpecialCharTok{==} \StringTok{"SA"}\NormalTok{, }\StringTok{"mean"}\NormalTok{],}
         \AttributeTok{y =}\NormalTok{ fenFlavTotal[fenFlavTotal}\SpecialCharTok{$}\NormalTok{Endulzante }\SpecialCharTok{==} \StringTok{"ST"}\NormalTok{, }\StringTok{"mean"}\NormalTok{] )}
\end{Highlighting}
\end{Shaded}

\begin{verbatim}
## 
##  F test to compare two variances
## 
## data:  fenFlavTotal[fenFlavTotal$Endulzante == "SA", "mean"] and fenFlavTotal[fenFlavTotal$Endulzante == "ST", "mean"]
## F = 1.376, num df = 63, denom df = 63, p-value = 0.208
## alternative hypothesis: true ratio of variances is not equal to 1
## 95 percent confidence interval:
##  0.8359404 2.2648859
## sample estimates:
## ratio of variances 
##           1.375976
\end{verbatim}

\begin{Shaded}
\begin{Highlighting}[]
\FunctionTok{var.test}\NormalTok{(}\AttributeTok{x =}\NormalTok{ fenFlavTotal[fenFlavTotal}\SpecialCharTok{$}\NormalTok{Endulzante }\SpecialCharTok{==} \StringTok{"ST"}\NormalTok{, }\StringTok{"mean"}\NormalTok{],}
         \AttributeTok{y =}\NormalTok{ fenFlavTotal[fenFlavTotal}\SpecialCharTok{$}\NormalTok{Endulzante }\SpecialCharTok{==} \StringTok{"SU"}\NormalTok{, }\StringTok{"mean"}\NormalTok{] )}
\end{Highlighting}
\end{Shaded}

\begin{verbatim}
## 
##  F test to compare two variances
## 
## data:  fenFlavTotal[fenFlavTotal$Endulzante == "ST", "mean"] and fenFlavTotal[fenFlavTotal$Endulzante == "SU", "mean"]
## F = 1.0822, num df = 63, denom df = 63, p-value = 0.7548
## alternative hypothesis: true ratio of variances is not equal to 1
## 95 percent confidence interval:
##  0.6574828 1.7813753
## sample estimates:
## ratio of variances 
##           1.082231
\end{verbatim}

\hypertarget{way-anova}{%
\subsubsection{2WAY-ANOVA}\label{way-anova}}

Dado que comparamos dos variables independientes cualitativas y una
variable dependiente cuantitativa

\hypertarget{box-plot}{%
\paragraph{Box-plot}\label{box-plot}}

\begin{Shaded}
\begin{Highlighting}[]
\FunctionTok{library}\NormalTok{(}\StringTok{"ggplot2"}\NormalTok{)}
\FunctionTok{library}\NormalTok{(}\StringTok{"gridExtra"}\NormalTok{)}

\FunctionTok{ggplot}\NormalTok{(}\AttributeTok{data =} \FunctionTok{as.data.frame}\NormalTok{(fenFlavTotal), }\FunctionTok{aes}\NormalTok{(}\AttributeTok{x =}\NormalTok{ Endulzante, }\AttributeTok{y =}\NormalTok{ mean, }\AttributeTok{color =}\NormalTok{ Endulzante)) }\SpecialCharTok{+}
  \FunctionTok{geom\_boxplot}\NormalTok{() }\SpecialCharTok{+}
  \FunctionTok{theme\_bw}\NormalTok{()}
\end{Highlighting}
\end{Shaded}

\includegraphics{dataMetabolitos_files/figure-latex/box-plot-1.pdf}

\begin{Shaded}
\begin{Highlighting}[]
\NormalTok{p1 }\OtherTok{\textless{}{-}} \FunctionTok{ggplot}\NormalTok{(}\AttributeTok{data =} \FunctionTok{as.data.frame}\NormalTok{(fenFlavTotal), }\FunctionTok{aes}\NormalTok{(}\AttributeTok{x =}\NormalTok{ Endulzante, }\AttributeTok{y =}\NormalTok{ mean)) }\SpecialCharTok{+} 
  \FunctionTok{geom\_boxplot}\NormalTok{() }\SpecialCharTok{+} \FunctionTok{theme\_bw}\NormalTok{()}
\NormalTok{p2 }\OtherTok{\textless{}{-}} \FunctionTok{ggplot}\NormalTok{(}\AttributeTok{data =} \FunctionTok{as.data.frame}\NormalTok{(fenFlavTotal), }\FunctionTok{aes}\NormalTok{(}\AttributeTok{x =}\NormalTok{ Condiciones, }\AttributeTok{y =}\NormalTok{ mean)) }\SpecialCharTok{+}
  \FunctionTok{geom\_boxplot}\NormalTok{() }\SpecialCharTok{+} \FunctionTok{theme\_bw}\NormalTok{()}
\NormalTok{p3 }\OtherTok{\textless{}{-}} \FunctionTok{ggplot}\NormalTok{(}\AttributeTok{data =} \FunctionTok{as.data.frame}\NormalTok{(fenFlavTotal), }\FunctionTok{aes}\NormalTok{(}\AttributeTok{x =}\NormalTok{ Endulzante, }\AttributeTok{y =}\NormalTok{ mean, }\AttributeTok{colour =}\NormalTok{ Condiciones)) }\SpecialCharTok{+}
  \FunctionTok{geom\_boxplot}\NormalTok{() }\SpecialCharTok{+} \FunctionTok{theme\_bw}\NormalTok{()}

\FunctionTok{grid.arrange}\NormalTok{(p1, p2, }\AttributeTok{ncol =} \DecValTok{2}\NormalTok{)}
\end{Highlighting}
\end{Shaded}

\includegraphics{dataMetabolitos_files/figure-latex/box-plot-2.pdf}

\begin{Shaded}
\begin{Highlighting}[]
\NormalTok{p3}
\end{Highlighting}
\end{Shaded}

\includegraphics{dataMetabolitos_files/figure-latex/box-plot-3.pdf}

\hypertarget{dudas}{%
\subsubsection{Dudas}\label{dudas}}

Hesperidina?

\hypertarget{muestras-bio}{%
\subsection{Muestras bio}\label{muestras-bio}}

\hypertarget{cronico-orina-atipicos}{%
\subsubsection{Cronico Orina Atipicos}\label{cronico-orina-atipicos}}

\hypertarget{cronico-plasma-atipicos}{%
\subsubsection{Cronico plasma Atipicos}\label{cronico-plasma-atipicos}}

\end{document}
